\documentclass[aspectratio=43]{beamer}

\usepackage[utf8]{inputenc} % mindenképp maradjon az utf-8 kódolás
\usepackage[magyar]{babel}
\usepackage[T1]{fontenc}
\usepackage{amsmath}
\usepackage{amsfonts}
\usepackage{amssymb}
\usepackage{graphicx} % grafikus elemek, képek berakásához
%\usepackage{blindtext}
%\usepackage{hyperref} % PDF hivatkozásokhoz kell
\usepackage[hang]{caption}
%\usepackage{xcolor}
%\usepackage[affil-it]{authblk}

%\definecolor{rosewood}{rgb}{0.6, 0.0, 0.04}
%\definecolor{indigo(dye)}{rgb}{0.0, 0.25, 0.42}

\usetheme{default}	% téma
\usecolortheme{seahorse}
\usefonttheme{serif}	% gyönyörű talpas betűtípus

\beamertemplatenavigationsymbolsempty % a pdf-be ágyazott navigációs gombok kikapcsolása

\title{Modell-redukció alkalmazása\\az elektromágneses térszámításban}			% cím
%\subtitle{\vspace{0.3cm}Proper Orthogonal Decomposition} 	% alcím

\date{\today}

\author{Szilágyi Gábor\\[3ex]Konzulens: Dr. Bilicz Sándor}	% szerző

%\institute{Institute} % intézmény vagy más infó a szerzőről
%\logo{\includegraphics[height=0.5cm]{bme_logo.pdf}}
%\newcommand{\pagenum}{\hfill\insertframenumber/\insertpresentationendpage\hspace{-\fill}}
\newcommand{\numframetitle}[1]{\frametitle{#1\hfill\insertframenumber/\insertpresentationendpage\hspace{-\fill}}}
\begin{document}
\maketitle	% címoldal
\begin{frame}
	\numframetitle{Az 1. dia címe}
	\framesubtitle{Az 1. dia alcíme}
	\begin{columns}	
		\column{0.48\textwidth}
		    asdasdasd
        \column{0.48\textwidth}
			Lehetségesek ilyen átalakulások ütközések hatására:
			\begin{align*}
				Xe^+ + e^- \quad \longleftrightarrow \quad Xe
			\end{align*}		
			A töltött részecskék tere és a külső elektromágneses tér együtt hat a töltött részecskék mozgására.
			\begin{align*}
				\Sigma Q = 0
			\end{align*}
	\end{columns}
\end{frame}
\begin{frame}
	\numframetitle{Az egydimenziós modell}
	\begin{columns}
		\column{0.48\textwidth}
			A következő egyszerűsítésekkel jutunk 3D-ből az 1D plazmához: \\
			\begin{itemize}
				\item A töltetlen $Xe$ részecskéket elhagyjuk
				\item Az ütközésektől eltekintünk
				\item Csak az elektronok mozgását vizsgáljuk
				\item Pontszerű részecskék helyett felületi töltéssűrűséggel rendelkező, az $x$ tengelyre merőleges lapok
				\item A pozitív töltésű $Xe^+$ ionokat helyhez kötött háttér-töltéssűrűségnek vesszük
			\end{itemize}
		\column{0.48\textwidth}
			\begin{itemize}
				\item A szimulációs tér egydimenziós és ciklikus, $x=0 \Longleftrightarrow x=N_g$
				\item A külső elektromos teret 0-nak vesszük
				\item A mégneses térnek nincs hatása 1D-ben
			\end{itemize}
	\end{columns}
\end{frame}
\begin{frame}
	\numframetitle{A Particle-Mesh módszer}
	\begin{columns}
		\column{0.48\textwidth}
			Egy kis random szöveg
		\column{0.48\textwidth}
			\begin{figure}
				\includegraphics[width=\textwidth]{bme_logo.pdf}
			\end{figure}
	\end{columns}
\end{frame}
\end{document}
