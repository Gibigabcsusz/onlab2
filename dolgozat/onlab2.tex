% pdf/a 
\begin{filecontents*}[overwrite]{\jobname.xmpdata}
    \Title{Önálló laboratórium 2 dolgozat}
    \Author{Szilágyi Gábor}
    \Language{hu-HU}
    \Subject{Modell-redukció alkalmazása az elektromágneses térszámításban}
    \Keywords{POD}
    \Publisher{Szilágyi Gábor}
\end{filecontents*}

\documentclass[a4paper,12pt,titlepage]{article}
\usepackage{ucs}
\usepackage[T1]{fontenc}
\usepackage[utf8]{inputenc}
\usepackage[magyar]{babel}
\usepackage{amsfonts}
\usepackage{amsmath,bm}
%\usepackage{amssymb}
\usepackage{graphicx}
%\usepackage[hang]{caption}
%\usepackage{subcaption}
%\usepackage{enumerate}
%\usepackage{psfrag}
\usepackage[left=20mm,right=20mm,top=20mm,bottom=25mm]{geometry}
%\usepackage[left=10mm,right=10mm,top=10mm,bottom=15mm]{geometry} %landscape
%\usepackage[hyphenbreaks]{breakurl}
%\usepackage[hyphens]{url}
%\usepackage{multirow}
%\usepackage{booktabs}
\usepackage{hyperref}
%\usepackage{listings}
%\usepackage{cite}
%\usepackage{csquotes}
%\usepackage[range-phrase=--, range-units=single]{siunitx}
\usepackage{xcolor}
\usepackage[a-3u]{pdfx}
\hypersetup{
	colorlinks,
%	linkcolor={red!50!black},
	linkcolor={black},
%	citecolor={blue!50!black},
	citecolor={black},
%	urlcolor={blue!80!black}
	urlcolor={blue!80!black}
}

\pagestyle{plain} 

\listfiles % a package-ek kilistázása a logba

\title{
    \centering
    \includegraphics[width=0.6\textwidth]{kep/bme_logo.pdf} \\
    \vspace{0.5cm}
    \large{\bf{Budapesti Műszaki és Gazdaságtudományi Egyetem}\\
    \bf{Villamosmérnöki és Informatikai Kar}\\
    \bf{Szélessávú Hírközlés és Villamosságtan Tanszék}}\\
    \vspace{4cm}
    \large{\bf{Önálló laboratórium 2 dolgozat}} \\
    \vspace{1cm}
    \Large{\bf{Modell-redukció alkalmazása az elektromágneses térszámításban}} \\
    \vspace{2cm}
}

%\parskip=10pt
%\parindent=0pt

\newcommand\adj[1]{#1^{\mathrm{H}}}

\author{Szilágyi Gábor \\\vspace{2cm}\\ Konzulens: Dr. Bilicz Sándor}
\date{Budapest, \today}


\begin{document}
    \maketitle
    \tableofcontents
    \section{Meglátások}
        Itt fogom leírni, hogy mit sikerült eddig felfognom az elméleti háttérből.
    \section{Bevezetés}
        \subsection{POD madártávlatból}
            A POD, vagyis a Proper Orthogonal Decomposition egy modellredukciós eljárás, ami egy adott adathalmaz reprezentálásához optimális bázist keres meg. Az eljárás által meghatározott $\Psi$ bázisban a bázisvektoroknak az a tulajdonsága, hogy a lehető legkevesebb bázisvektorral leírható az adathalmaz információtartalmának vagy energiájának lehető legnagyobb része. Ezt felhasználva a $\Psi$ csonkolásával egy közelítő bázist lehet előállítani ($\Psi'$), ami lényegesen kisebb rendű, mint $\Psi$, mégis kis hibával reprezentálható benne az eredeti adathalmaz. Természetesen minél több bázisvektort hagyunk meg $\Psi'$-ben, annál jobban csökken a modell-redukcióból származó hiba, de a csonkolás mértékét az adott alkalmazáshoz mérten előírhatjuk. A POD egy másik előnyös tulajdonsága, hogy a gyakorlati esetek nagy részében a sorbarendezett $\psi_n$ bázisvektorokra eső energiatartalom rohamosan csökken, ezért sokszor nagyságrendekkel kisebb dimenziószámú bázissal is jól leírható az adathalmaz, mint az eredeti esetben.
        \subsection{Felhasználási területek}
            A POD eljárást számos tudományterületen sikeresen alkalmazták már.  Ezeknél a problémáknál az okozza általában a fő gondot, hogy a szimulált rendszer szabadsági fokainak száma nagyon nagy, emiatt egy-egy szimuláció nagyon sok ideig tart, pontatlanabb diszkretizált modell pedig fals eredményekre vezet. Lényeges felhasználási területek például: turbulens áramlások szimulációja; statisztikában az adathalmazok redukálása; szabályozástechnikában a szuboptimális, de gyorsan számítható beavatkozás.
        \subsection{Alkalmazás az elektromágneses térszámításban}
            Az EM térszámításban többféle kontextusban is hasznos lehet a POD a futási idő  vagy a memóriafelhasználás jelentős csökkentésére. Az egyik megközelítésben egy végeselem modellben zajló tranziens folyamat lefolyására kaphatunk számítás szempontjából olcsó, közelítő megoldást. Ehhez először a teljes kérdéses időintervallum első töredék részére egy teljes értékű szimulációt futtatunk, amely viszonylag sok számítást igényel. Ennek a rövid részmegoldásnak az eredményei szolgálnak a POD bemenetéül. A POD ezek alapján meghatározza a rendszer dinamikájában megjelenő struktúrákat, majd csak a lényeges összetevőkre szorítkozva egy lecsökkentett szabadsági fokú rendszert szimulálunk tovább a hátralévő időben, ami már fajlagosan kevesebb számítást igényel.
        \subsection{Egyéb megközelítések a POD használatához}
            A fent vázolt, tranziens szimulációban történő alkalmazáson kívül más módokon is hasznosítható lehet a POD a térszámítási problémákban.
            \par
            Optimalizálási feladatoknál szokott előfordulni az a probléma, hogy az optimalizációhoz használható paraméterek miatt exponenciálisan megnövekszik egy modell szabadsági fokainak a száma.
        \subsection{A felbontás egyenletei}
            \begin{align}
                {\bf S}~=&~{\bf V \Sigma \adj{U}} \\
                {\bf C}~=&~{\bf \adj{S} S} \\
                {\bf C}~=&~{\bf U \Sigma \adj{V} V\Sigma \adj{U}} \\
                {\bf C}~=&~{\bf U \Sigma}^2{\bf \adj{U}}
            \end{align}
            Itt ${\bf S}$ a snapshot-mátrix, ${\bf V}$ oszlopai az új bázisvektorok, ${\bf \Sigma}$ tartalmazza a bázisvektorok információtartalmát jellemző szinguláris értékeket a főátlójában, ${\bf \adj{U}}$ sorai az egyes bázisvektorok időfüggő együtthatói, ${\bf C}$ pedig a snapshotokból álló adathalmaz kovarianciamátrixa.
    \section{\LaTeX~Próba}
        Lorem ipsum \cite{PGD}.
        \begin{align}
            {\bf X}~=&~{\bf V\Sigma U}^*
        \end{align}
    \bibliography{mybib}
    \bibliographystyle{plain}
\end{document}

%\cite{Henneron14}

%            \begin{figure}
%                \centering
%                \includegraphics[width=0.8\textwidth]{kep/szerkesztett/wstk-mighty-gecko-nagy.jpg}
%                \caption{WSTK + radio board.}
%                \label{fig:wstkmighty}
%            \end{figure}
% \cite{Anritsu}
%            \begin{figure}
%                \centering
%                \begin{subfigure}{0.48\textwidth}
%                    \includegraphics[width=\textwidth]{kep/szerkesztett/sol-868-conducted.png}
%                    \caption{\SI{868}{MHz}}
%                \end{subfigure}
%                \begin{subfigure}{0.48\textwidth}
%                    \includegraphics[width=\textwidth]{kep/szerkesztett/sol-470-conducted.png}
%                    \caption{\SI{470}{MHz}}
%                \end{subfigure}
%                \caption{470 és \SI{868}{MHz}-es Sol radio board-ok kimeneti spektruma.}
%                \label{fig:sol-conducted}
%            \end{figure}
 
