% pdf/a 
\begin{filecontents*}[overwrite]{\jobname.xmpdata}
    \Title{Önálló laboratórium 2 dolgozat}
    \Author{Szilágyi Gábor}
    \Language{hu-HU}
    \Subject{Modell-redukció alkalmazása az elektromágneses térszámításban}
    \Keywords{POD}
    \Publisher{Szilágyi Gábor}
\end{filecontents*}

\documentclass[a4paper,12pt,titlepage]{article}
\usepackage{ucs}
\usepackage[T1]{fontenc}
\usepackage[utf8]{inputenc}
\usepackage[magyar]{babel}
\usepackage{amsfonts}
\usepackage{amsmath}
%\usepackage{amssymb}
\usepackage{graphicx}
%\usepackage[hang]{caption}
%\usepackage{subcaption}
%\usepackage{enumerate}
%\usepackage{psfrag}
\usepackage[left=20mm,right=20mm,top=20mm,bottom=25mm]{geometry}
%\usepackage[left=10mm,right=10mm,top=10mm,bottom=15mm]{geometry} %landscape
%\usepackage[hyphenbreaks]{breakurl}
%\usepackage[hyphens]{url}
%\usepackage{multirow}
%\usepackage{booktabs}
\usepackage{hyperref}
%\usepackage{listings}
\usepackage{cite}
%\usepackage{csquotes}
%\usepackage[range-phrase=--, range-units=single]{siunitx}
\usepackage{xcolor}
\usepackage[a-3u]{pdfx}
\hypersetup{
	colorlinks,
%	linkcolor={red!50!black},
	linkcolor={black},
%	citecolor={blue!50!black},
	citecolor={black},
%	urlcolor={blue!80!black}
	urlcolor={blue!80!black}
}

\pagestyle{plain} 

\listfiles % a package-ek kilistázása a logba

\parskip=10pt
\parindent=0pt

\author{Szilágyi Gábor \\\vspace{2cm}\\ Konzulens: Dr. Bilicz Sándor}
\date{Budapest, \today}


\begin{document}
    \begin{center}
        \huge Összefoglaló J. Nathan Kutz POD-ról szóló előadásairól
    \end{center}
    \section*{1. videó}
        \textbf{Def.:} $N \in \mathbb{N}$ --- a térbeli mintavételi vagy diszkretizálási pontok száma, a természetes bázis számossága (nagy) \par
        \textbf{Def.:} $M \in \mathbb{N}, M \ll N$ --- a redukált bázis számossága \par
        \textbf{Def.:} $u(x,t) = u \in \mathbb{C}$ --- a keresett függvény, ami idő- és helyfüggő \par
        \textbf{Megj.:} $u(x,t) = \displaystyle\sum_{i} a_i \cdot \phi_i$ --- a keresett függvény bázisfüggvényekkel való felírása \par
        \textbf{Def.:} $\phi_i(x) = \phi_i$ --- az $i$-edik bázisfüggvény \par
        \textbf{Def.:} $a_i(t) = a_i \in \mathbb{C}$ --- az $i$-edik bázisfüggvény együtthatója\par
        \subsection*{Bevezetés}
            Deriváltakból hogyan lesz véges differencia, diffegyenletekből hogyan lesznek véges differencia egyenletek (egydimenziós eset, $x \in [-L, L] \subset \mathbb{R}$, $x_i$ az $i$-edik térbeli mintavételi pont, $\Delta x = x_{i+1} - x_i$) \\
            \begin{align*}
                \frac{d u}{d x} \approx \frac{x_{i+1}-x_{i-1}}{2\cdot \Delta x}, \quad
                \frac{d^2 u}{d x^2} \approx \frac{x_{i-1}-2\cdot x_{i}+x_{i+1}}{\Delta x^2}
            \end{align*}
        \subsection*{Változók szétválasztása} Erre épül a POD azzal, hogy a megválasztott bázisfüggvényekből (amelyek a térkoordináták függvényei) és a bázisfüggvények együtthatóiból (amelyek időfüggőek) állítjuk elő a differenciálegyenlet-rendszer megoldását. Diszkrét esetben, tehát a szimulációk során is, a bázisfüggvények térbeli mintavételezésével kapjuk a bázisvektorokat és a bázisfüggvények együtthatóinak időbeli mintavételezésével kapjuk a bázisvektorok együtthatóit.
\end{document}


%            \begin{figure}
%                \centering
%                \includegraphics[width=0.8\textwidth]{kep/szerkesztett/wstk-mighty-gecko-nagy.jpg}
%                \caption{WSTK + radio board.}
%                \label{fig:wstkmighty}
%            \end{figure}
% \cite{Anritsu}
%            \begin{figure}
%                \centering
%                \begin{subfigure}{0.48\textwidth}
%                    \includegraphics[width=\textwidth]{kep/szerkesztett/sol-868-conducted.png}
%                    \caption{\SI{868}{MHz}}
%                \end{subfigure}
%                \begin{subfigure}{0.48\textwidth}
%                    \includegraphics[width=\textwidth]{kep/szerkesztett/sol-470-conducted.png}
%                    \caption{\SI{470}{MHz}}
%                \end{subfigure}
%                \caption{470 és \SI{868}{MHz}-es Sol radio board-ok kimeneti spektruma.}
%                \label{fig:sol-conducted}
%            \end{figure}
 
