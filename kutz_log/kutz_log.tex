% pdf/a 
\begin{filecontents*}[overwrite]{\jobname.xmpdata}
    \Title{Önálló laboratórium 2 dolgozat}
    \Author{Szilágyi Gábor}
    \Language{hu-HU}
    \Subject{Modell-redukció alkalmazása az elektromágneses térszámításban}
    \Keywords{POD}
    \Publisher{Szilágyi Gábor}
\end{filecontents*}

\documentclass[a4paper,12pt,titlepage]{article}
\usepackage{ucs}
\usepackage[T1]{fontenc}
\usepackage[utf8]{inputenc}
\usepackage[magyar]{babel}
\usepackage{amsfonts}
\usepackage{amsmath}
%\usepackage{amssymb}
\usepackage{graphicx}
%\usepackage[hang]{caption}
%\usepackage{subcaption}
%\usepackage{enumerate}
%\usepackage{psfrag}
\usepackage[left=20mm,right=20mm,top=20mm,bottom=25mm]{geometry}
%\usepackage[left=10mm,right=10mm,top=10mm,bottom=15mm]{geometry} %landscape
%\usepackage[hyphenbreaks]{breakurl}
%\usepackage[hyphens]{url}
%\usepackage{multirow}
%\usepackage{booktabs}
\usepackage{hyperref}
%\usepackage{listings}
\usepackage{cite}
%\usepackage{csquotes}
%\usepackage[range-phrase=--, range-units=single]{siunitx}
\usepackage{xcolor}
\usepackage[a-3u]{pdfx}
\hypersetup{
	colorlinks,
%	linkcolor={red!50!black},
	linkcolor={black},
%	citecolor={blue!50!black},
	citecolor={black},
%	urlcolor={blue!80!black}
	urlcolor={blue!80!black}
}

\pagestyle{plain} 

\listfiles % a package-ek kilistázása a logba

\parskip=10pt
\parindent=0pt

\author{Szilágyi Gábor \\\vspace{2cm}\\ Konzulens: Dr. Bilicz Sándor}
\date{Budapest, \today}


\begin{document}
    \begin{center}
        \huge Összefoglaló J. Nathan Kutz POD-ról szóló előadásairól
    \end{center}
        Az előadó 3, egyenként szűk egy órás videóban mutatja be a POD, vagyis a Proper Orthogonal Decomposition módszer alapjait. A videók elérhetőek az előadó honlapján: \url{https://faculty.washington.edu/kutz/rom/page1/page5/rom.html} vagy YouTube-on: \url{https://www.youtube.com/c/NathanKutzAMATH} (POD introduction 1,2,3) \\
        Linkek elérve: 2022.10.08.
    \section*{1. videó}
        \subsection*{Néhány jelölés és definíció}
            \textbf{Def.:} $N \in \mathbb{N}$ --- a térbeli mintavételi vagy diszkretizálási pontok száma, a természetes bázis számossága (nagy) \par
            \textbf{Def.:} $N \in \mathbb{N}$ --- a térbeli mintavételi vagy diszkretizálási pontok száma, a természetes bázis számossága (nagy) \par
            \textbf{Def.:} $M \in \mathbb{N}, M \ll N$ --- a redukált bázis számossága \par
            \textbf{Def.:} $u(x,t) = u \in \mathbb{C}$ --- a keresett függvény, ami idő- és helyfüggő \par
            \textbf{Def.:} $\varPsi_i(x) = \varPsi_i$ --- az $i$-edik bázisfüggvény \par
            \textbf{Def.:} $a_i(t) = a_i \in \mathbb{C}$ --- az $i$-edik bázisfüggvény együtthatója\par
            \textbf{Megj.:} $u(x,t) = \displaystyle\sum_{i} a_i \cdot \varPsi_i$ --- a keresett függvény bázisfüggvényekkel való felírása \par
        \subsection*{Diszkretizálás}
            Deriváltakból hogyan lesz véges differencia, diffegyenletekből hogyan lesznek véges differencia egyenletek (egydimenziós eset, $x \in [-L, L] \subset \mathbb{R}$, $x_n$ az $n$-edik térbeli mintavételi pont, $\Delta x = x_{n+1} - x_n$). Diszkrét esetben, tehát a szimulációk során is, a bázisfüggvények térbeli mintavételezésével kapjuk a bázisvektorokat és a bázisfüggvények együtthatóinak időbeli mintavételezésével kapjuk a bázisvektorok együtthatóit. ($u_n$ az $u$ függvény $n$-edik mintavételi pontban felvett értéke, de $u_t$, $u_x$, ... az $u$ parciális deriváltjai $t$, $x$, ... szerint. $\mu$ valamilyen paraméter.) \\
            \begin{align*}
                \frac{d u}{d x} \approx \frac{x_{n+1}-x_{n-1}}{2\cdot \Delta x}, \quad
                \frac{d^2 u}{d x^2} \approx \frac{x_{n-1}-2\cdot x_{n}+x_{n+1}}{\Delta x^2}, \quad \dots
            \end{align*}
            \begin{align*}
                u(x,t) \quad \longrightarrow & \quad u(x_n,t) = u_n, \quad n = 1,2,3,...,N \\
                \frac{\partial u}{\partial t} = u_t = F(u,u_x,u_{xx},...,t,\mu) \quad \longrightarrow & \quad
                \frac{d u_n}{d t} = F(u_n,u_{n\pm1},u_{n\pm2},...,t,\mu)
            \end{align*}
        \subsection*{Változók szétválasztása}
            Erre épül a POD azzal, hogy a megválasztott bázisfüggvényekből ($\varPsi_i$, amelyek a térkoordináták függvényei) és a bázisfüggvények együtthatóiból ($a_i$, amelyek időfüggőek) állítjuk elő a megoldást, $u$-t. A változók szétválasztása nem minden esetben tehető meg formálisan, de megfelelő számú bázisvektor használata esetén ilyenkor is lehet jó eredményre jutni a numerikus módszereknél. Feltesszük, hogy a változók ($x$ és $t$) szétválasztása megtehető, emiatt:
            \begin{align*}
                \frac{\partial u}{\partial t}~= &~F(u,~u_x,~u_{xx},~...~,~t,~\mu) \\
                \Downarrow & \\
                \sum \frac{d a_{i}}{d t} \varPsi_{i}~= &~F\left(\sum a_{i} \varPsi_{i},~\sum a_{i} \frac{d\varPsi_{i}}{dx},~\sum a_{i} \frac{d^2\varPsi_{i}}{dx^2},~...~,~t,~\mu\right)
            \end{align*}
            A jobb kezelhetőség érdekében ortonormált bázist választunk:
            \begin{align*}
                \left( \varPsi_{i},\varPsi_{j}  \right)~=~\int_{-L}^{L}\varPsi_{i} \varPsi_{j}  ~dx~=~\delta_{ij}~=~
                \begin{cases}
                    0 & i \ne j \\
                    1 & i = j
                \end{cases}
            \end{align*}
            A $\varPsi$ bázis ortonormalitása miatt a következővé alakítható a megoldásfüggvény idő szerinti deriváltjának fenti egyenlete:
            \begin{align*}
                \sum \frac{d a_{i}}{d t} \varPsi_{i}~=&~F\left(\sum a_{i} \varPsi_{i},~\sum a_{i} \frac{d\varPsi_{i}}{dx},~\sum a_{i} \frac{d^2\varPsi_{i}}{dx^2},~...~,~t,~\mu\right) \\
                \Downarrow & \quad \int_{-L}^{L} (\phantom{abc})\cdot \varPsi_{j}~dx \\
                \frac{d a_{j}}{d t}~=&~\left( F, \varPsi_{j} \right), \quad j = 1,2,...,N
            \end{align*}
            A kapott kifejezés magában nem tűnik előrelépésnek, de a megfelelő $\varPsi$ bázis megválasztásával nem kell mind az $N$ bázisvektort figyelembe venni, hanem csak $M$-et közülük, és így is egy jó közelítő megoldást kapunk, mert a többi bázisvektor együtthatója numerikusan inszignifikáns.
            \subsection*{MATLAB példa}
                A következőkben az előadó bemutatja a Fourier-transzformáció esetén, hogy az adott problémától és annak paramétereitől függően nem mindig járunk jól, ha egy előre meghatározott bázisra képezzük le a problémát. A diszkrét Fourier-transzformáció numerikusan nagyon hatékony a használt bázis számosságához képest és egyéb kedvező tulajdonságai is vannak. Ha viszont olyan problémával kell megküzdeni, amihez nem jól illeszkednek a fourier-módusok és nagy a szabadsági fokainak száma, akkor sokkal jobban járhatunk, ha az adott problémához igazított optimális bázist választunk. A POD-nak pedig éppen ez a bázisválasztás a célja.
    \newpage
        \section*{2. Videó}
\end{document}


%            \begin{figure}
%                \centering
%                \includegraphics[width=0.8\textwidth]{kep/szerkesztett/wstk-mighty-gecko-nagy.jpg}
%                \caption{WSTK + radio board.}
%                \label{fig:wstkmighty}
%            \end{figure}
% \cite{Anritsu}
%            \begin{figure}
%                \centering
%                \begin{subfigure}{0.48\textwidth}
%                    \includegraphics[width=\textwidth]{kep/szerkesztett/sol-868-conducted.png}
%                    \caption{\SI{868}{MHz}}
%                \end{subfigure}
%                \begin{subfigure}{0.48\textwidth}
%                    \includegraphics[width=\textwidth]{kep/szerkesztett/sol-470-conducted.png}
%                    \caption{\SI{470}{MHz}}
%                \end{subfigure}
%                \caption{470 és \SI{868}{MHz}-es Sol radio board-ok kimeneti spektruma.}
%                \label{fig:sol-conducted}
%            \end{figure}
 
