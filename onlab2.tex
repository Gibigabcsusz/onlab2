% pdf/a 
\begin{filecontents*}[overwrite]{\jobname.xmpdata}
    \Title{Önálló laboratórium 2 dolgozat}
    \Author{Szilágyi Gábor}
    \Language{hu-HU}
    \Subject{Modell-redukció alkalmazása az elektromágneses térszámításban}
    \Keywords{POD}
    \Publisher{Szilágyi Gábor}
\end{filecontents*}

\documentclass[a4paper,12pt,titlepage]{article}
\usepackage{ucs}
\usepackage[T1]{fontenc}
\usepackage[utf8]{inputenc}
\usepackage[magyar]{babel}
\usepackage{amsfonts}
\usepackage{amsmath}
%\usepackage{amssymb}
\usepackage{graphicx}
%\usepackage[hang]{caption}
%\usepackage{subcaption}
%\usepackage{enumerate}
%\usepackage{psfrag}
\usepackage[left=20mm,right=20mm,top=20mm,bottom=25mm]{geometry}
%\usepackage[left=10mm,right=10mm,top=10mm,bottom=15mm]{geometry} %landscape
%\usepackage[hyphenbreaks]{breakurl}
%\usepackage[hyphens]{url}
%\usepackage{multirow}
%\usepackage{booktabs}
\usepackage{hyperref}
%\usepackage{listings}
\usepackage{cite}
%\usepackage{csquotes}
%\usepackage[range-phrase=--, range-units=single]{siunitx}
\usepackage{xcolor}
\usepackage[a-3u]{pdfx}
\hypersetup{
	colorlinks,
%	linkcolor={red!50!black},
	linkcolor={black},
%	citecolor={blue!50!black},
	citecolor={black},
%	urlcolor={blue!80!black}
	urlcolor={blue!80!black}
}

\pagestyle{plain} 

\listfiles % a package-ek kilistázása a logba

\title{
    \centering
    \includegraphics[width=0.6\textwidth]{kep/bme_logo.pdf} \\
    \vspace{0.5cm}
    \large{\textbf{Budapesti Műszaki és Gazdaságtudományi Egyetem}\\
    \textbf{Villamosmérnöki és Informatikai Kar}\\
    \textbf{Szélessávú Hírközlés és Villamosságtan Tanszék}}\\
    \vspace{5cm}
    \huge{\textbf{Önálló laboratórium 2 dolgozat}} \\
    \vspace{3cm}
}

%\parskip=10pt
%\parindent=0pt

\author{Szilágyi Gábor \\\vspace{2cm}\\ Konzulens: Dr. Bilicz Sándor}
\date{Budapest, \today}


\begin{document}
    \maketitle
    \tableofcontents
    \section{Meglátások}
        Itt fogom leírni, hogy mit sikerült eddig felfognom az elméleti háttérből.
        \subsection{POD madártávlatból}
            Nemlineáris problémák esetén szokott adódni olyan megoldandó, $M$-ismeretlenes egyenletrendszer, ahol $M$ a szimulált tér diszkretizált pontjainak a száma ($M \gg 1$). Az eredeti egyenletrendszer egzakt megoldása helyett egy jelentősen csökkentett $N$ dimenziójú ($N \ll M$) egyenletrendszert oldunk meg, aminek a megoldása jól közelíti az eredetiét.
    \section{\LaTeX~próba}
            Lorem ipsum \cite{PGD}
    \bibliography{mybib}
    \bibliographystyle{plain}
    \addcontentsline{toc}{section}{Hivatkozások}
\end{document}


%            \begin{figure}
%                \centering
%                \includegraphics[width=0.8\textwidth]{kep/szerkesztett/wstk-mighty-gecko-nagy.jpg}
%                \caption{WSTK + radio board.}
%                \label{fig:wstkmighty}
%            \end{figure}
% \cite{Anritsu}
%            \begin{figure}
%                \centering
%                \begin{subfigure}{0.48\textwidth}
%                    \includegraphics[width=\textwidth]{kep/szerkesztett/sol-868-conducted.png}
%                    \caption{\SI{868}{MHz}}
%                \end{subfigure}
%                \begin{subfigure}{0.48\textwidth}
%                    \includegraphics[width=\textwidth]{kep/szerkesztett/sol-470-conducted.png}
%                    \caption{\SI{470}{MHz}}
%                \end{subfigure}
%                \caption{470 és \SI{868}{MHz}-es Sol radio board-ok kimeneti spektruma.}
%                \label{fig:sol-conducted}
%            \end{figure}
 
